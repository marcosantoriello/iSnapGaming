\documentclass[12pt, a4paper, oneside]{book}
\usepackage{fontspec}
%\setmainfont{Arial}[ItalicFont={Arial Italic}]
%\setmainfont{Gill Sans MT}[ItalicFont={Gill Sans MT Italic}]
\usepackage[utf8]{inputenc}
\usepackage[margin=1.5cm, bindingoffset=1cm]{geometry}
\linespread{1.5}
\usepackage{float}
\usepackage{multicol}
\usepackage{csquotes}
\usepackage{subfig}
\usepackage{graphicx}
\usepackage{wrapfig}
\usepackage[table]{xcolor}
\usepackage{indentfirst}
\setlength{\parindent}{0cm}
\usepackage[italian]{babel}
\usepackage{amsmath,amssymb}
\usepackage{hyperref}
\usepackage{color}
\usepackage{listings}
\usepackage{wrapfig}
\usepackage{url}
\lstset{showstringspaces=false}

%TABELLE
\usepackage{booktabs} % Per le linee orizzontali di alta qualità
\usepackage{tabularx} % Per tabelle di larghezza fissa con colonne espandibili
\definecolor{headercolor}{gray}{0.9} % Imposta il colore dell'intestazione a grigio
\usepackage{array}
\setlength{\arrayrulewidth}{0.3mm}

\usepackage{eurosym} %simbolo Euro

\usepackage{fancyhdr}
\pagestyle{fancy}
\renewcommand{\chaptermark}[1]{%
    \markboth{\chaptername
    \ \thechapter.\ #1}{}}
\fancyhf{}
\fancyhead[L]{\textsl{\leftmark}}
\fancyhead[R]{\textsl{IS- Problem Statement}}
\fancyfoot[C]{\thepage}


\title{Problem Statement - IS}
\author{Chiara Coscarelli, Angelo Fiorillo, Marco Santoriello, Fabio Sessa}
\date{Ottobre 2023}

\begin{document}
    \begin{titlepage}
        \begin{center}
            \LARGE{\uppercase{Università degli Studi di Salerno}}\\
            \vspace{5mm}
            %Dipartimento
            \uppercase{\normalsize Dipartimento di Informatica}\\
        \end{center}
        \begin{figure}[H]
            \centering
            \includegraphics[width=0.3\textwidth]{img/logo_unisa.png}
        \end{figure}

        \begin{center}
            %Corso di Laurea
            \normalsize{ Corso di Laurea in informatica }\\
            \vspace{5mm}
            %Titolo
            {\LARGE{\bf Problem Statement - iSnapGaming }}\\
            \vspace{10mm}
        \end{center}

        \vspace{15mm}
        \noindent
        \begin{minipage}[t]{0.47\textwidth}
        %Relatore
        {\large{ Docente:\\\bf  Prof. Andrea De Lucia}}
            \vspace{12mm}\\
        \end{minipage}
        \hfill
        \begin{minipage}[t]{0.4\textwidth}\raggedright
        %Candidato
        {\large{Componenti: \\ \bf Chiara Coscarelli\\Angelo Fiorillo\\Marco Santoriello\\Fabio Sessa}}
        \end{minipage}

        \vspace{20mm}

        \vspace{50mm}
        %Anno Accademico
        \centering{\large \uppercase{ Anno Accademico 2023/2024 }}

    \end{titlepage}
    \newpage
%MEMBRI DEL PROGETTO
    \begin{center}
    {\LARGE{Membri del progetto}}
    \end{center}
    \begin{center}
        \begin{tabular}{|c|c|m{8cm}|}
            \hline
            \cellcolor{lightgray} \textbf{Nome} & \cellcolor{lightgray} \textbf{Matricola} & \cellcolor{lightgray} \textbf{E-mail}\\ \hline
            Coscarelli Chiara (CC) & 0512113869 & c.coscarelli1@studenti.unisa.it\\ \hline
            Fiorillo Angelo (AF) & 0512105739 & a.fiorillo9@studenti.unisa.it\\ \hline
            Santoriello Marco (MS) &  0512114100 & m.santoriello37@studenti.unisa.it\\ \hline
            Sessa Fabio (FS) & 0512114634 & f.sessa30@studenti.unisa.it\\ \hline
        \end{tabular}
    \end{center}

    \begin{center}
    {\LARGE{Revision History}}
    \end{center}

    %TABELLA REVISION HISTORY
    \begin{center}
        \begin{tabular}{|c|c|m{8cm}|c|}
            \hline
            \cellcolor{lightgray} \textbf{Data} & \cellcolor{lightgray} \textbf{Versione} & \cellcolor{lightgray} \textbf{Descrizione} & \cellcolor{lightgray} \textbf{Autore}\\ \hline
            09/10/2023 & 0.00 & Impostazione documento e descrizione del problema & MS\\ \hline
            10/10/2023 & 0.01 & Aggiunta scenari S04, S06, S08, S09, S10 & MS\\ \hline
            11/10/2023 & 0.02 & Aggiunta scenari S03, S05, S12, S13 & FS\\ \hline
            11/10/2023 & 0.03 & Aggiunta scenari S01, S02, S14 & CC\\ \hline
            11/10/2023 & 0.04 & Aggiunta scenari S07, S16 & AF\\ \hline
            15/10/2023 & 0.05 & Aggiunta scenario S11 & AF\\ \hline
            15/10/2023 & 0.06 & Modifiche di allineamento a scelte comuni & Tutti\\ \hline
            20/10/2023 & 0.07 & Aggiunta Requisiti non funzionali & Tutti\\ \hline
            24/10/2023 & 0.08 & Aggiunta Requisiti funzionali & Tutti\\ \hline
            24/10/2023 & 0.09 & Revisione Requisiti non funzionali & Tutti\\ \hline
            24/10/2023 & 0.10 & Ordinamento scenari & MS\\ \hline
        \end{tabular}
    \end{center}





    \setcounter{tocdepth}{3} %IMPOSTO LIVELLO PROFONDITA' INDICE

    \tableofcontents

    \chapter*{Problem Statement}
    %La seguente istruzione aggiunge il capitolo all'indice, siccome il comando
    %con la star * non inserisce il numero del capitolo/sezione ma non mostra
    %%nemmeno la sezione nell'indice
    \addcontentsline{toc}{chapter}{\protect\numberline{}Problem Statement}
    \section*{Problem}
    \addcontentsline{toc}{section}{\protect\numberline{}Problem}
        iSnapGaming nasce come un e-commerce specializzato nella vendita di videogiochi. Non ha una sede fisica, dunque l'unica entrata del business è rappresentata esclusivamente dalla piattaforma di vendite online.

    \section*{Scenarios}
    \addcontentsline{toc}{section}{\protect\numberline{}Scenarios}
        \subsection*{S01: Registrazione}
    \addcontentsline{toc}{subsection}{\protect\numberline{}S01: Registrazione}
         L'utente Luca Neri, una volta entrato sulla piattaforma, accede al suo account utilizzando l'apposita funzionalità di "login", che lo
         reindirizza alla pagina di login del sito.
         Una volta qui, l'utente seleziona "Registrati Ora". Fatto ciò, viene immediatamente reindirizzato alla pagina di registrazione.
         Su questa pagina, Luca compila attentamente tutti i campi necessari. Nei campi obbligatori, inserisce le seguenti informazioni:
         Nome: Luca, Cognome: Neri, Data di nascita: 27/05/1998, Codice Fiscale: LCUNRE98E27F205X,
         Username: nlucas98, Indirizzo e-mail: lneri98@gmail.com e Password: LaMiaPasswordSafe1!.
         La registrazione termina con successo, e l’utente viene riportato sulla Home page.

    \subsection*{S02: Login Cliente}
    \addcontentsline{toc}{subsection}{\protect\numberline{}S02: Login Cliente}
        Il cliente Mario Rossi, che si trova nella pagina del catalogo, accede alla pagina di Login selezionando l'apposita funzionalità "Login".
        Su questa pagina, Mario inserisce le sue credenziali: username: marrossi, password: mypw123 e conferma l'accesso.
        L'autenticazione va a buon fine e il cliente viene reindirizzato alla pagina del catalogo,dove aveva interrotto.

    \subsection*{S03: Aggiunta di un prodotto}
    \addcontentsline{toc}{subsection}{\protect\numberline{}S03: Aggiunta di un prodotto}
        Il gestore del catalogo Marco Bianchi deve aggiungere il gioco Cyberpunk 2077 al catalogo. Il gestore
        effettua l'accesso selezionando la sezione di autenticazione dove inserisce i seguenti dati: username: mbianchi, Password: Mari0bianchi
        e dopodiché spunta il ruolo di cui fa parte, ossia "Gestore del catalogo".
        Il sistema avverte il gestore che l'autenticazione è andata a buon fine. Il gestore accede alla sezione
        Aggiungi Prodotto che lo porta alla schermata dedicata. Il gestore inserisce le seguenti informazioni:
        ID: x123P, Nome: Cyberpunk 2077, Produttore: CD Projekt RED, Piattaforma: PS5, Prezzo: 49.99, Quantità: 250,
        Categoria: Action RPG, PEGI: 18, Anno di Produtzione: 2020. Il gestore invia le informazioni
        e il sistema avverte che l'operazione di aggiunta è stata eseguita con successo.

    \subsection*{S04: Modifica di un prodotto}
    \addcontentsline{toc}{subsection}{\protect\numberline{}S04: Modifica di un prodotto}
        Il \textit{gestore dei prodotti} Michele Verdi entra nella piattaforma, si autentica attraverso la schermata di login, assicurandosi
        che l'opzione riportante il ruolo \textit{Gestore Catalogo} sia selezionata, ed accede alla pagina del catalogo. Da qui, entra
        nella pagina del videogioco \textit{Assassin's Creed: Valhalla}, e clicca sull'opzione che consente di modificare il prodotto, dato che vuole farlo apparire
        come \textit{non disponibile}, a causa del ritardo del fornitore che avrebbe dovuto consegnargli le nuove copie. Seleziona, quindi, l'opzione \textit{Disponibile},
        e la imposta su \textit{No}. Clicca sull'opzione di salvataggio delle modifiche, e viene reindirizzato alla pagina del prodotto, che presenta la modifica appena effettuata.


    \subsection*{S05: Rimozione di un prodotto}
    \addcontentsline{toc}{subsection}{\protect\numberline{}S05: Rimozione di un prodotto}
        Il gestore del catalogo Marco Bianchi deve rimuovere il gioco Cyberpunk 2077 dal catalogo.
        Il gestore effettua l'accesso selezionando la sezione di autenticazione dove inserisce i dati: username: mbianchi, Password: Mari0bianchi
        e dopodiché spunta il ruolo di cui fa parte, ossia "Gestore del catalogo".
        Il sistema avverte il gestore che l'autenticazione è andata a buon fine. Il gestore accede alla sezione del Catalogo.
        Il gestore utilizza la funzionalità della ricerca per categoria dei prodotti e seleziona "Action RPG". Il gestore
        trova Cyberpunk 2077 e accede all'area riservata ai dettagli. Il gestore seleziona la funzionalità che lo
        permette di eliminare definitivamente il prodotto dal catalogo. Il sistema avverte il gestore che l'operazione è avvenuta con successo.

    \subsection*{S06: Aggiunta di un nuovo gestore degli ordini}
    \addcontentsline{toc}{subsection}{\protect\numberline{}S06: Aggiunta di un nuovo gestore degli ordini}
        L'\textit{amministratore dei gestori} Francesco Bianchi, dalla pagina principale, accede al suo account e viene rimandato sulla HomePage.
        Da qui, seleziona l'opzione di aggiunta di un nouvo gestore da un menu dedicato. Viene così portato sulla pagina che permette di inserire un nuovo utente.
        Compila i vari campi: \textit{Nome: Antonio}, \textit{Cognome: Romano}, \textit{Data di nascita: 15/03/2000}, \textit{Codice Fiscale: RMNNTN00C15H703B},
        \textit{Username: goRomano}, \textit{Indirizzo e-mail: aRomano27@gmail.com} e \textit{Password: LaTuaPassword1}, seleziona il ruolo \textit{Gestore Ordini} tra le
        varie opzioni presernti e, infine, seleziona la funzionalità che permette di confermare l'aggiunta. Gli appare, quindi, un messaggio che lo informa circa il successo
        dell'operazione.

    \subsection*{S07: Ricerca prodotti}
    \addcontentsline{toc}{subsection}{\protect\numberline{}S07: Ricerca prodotti}
        Il cliente Francesco accede alla homepage iSnapGaming, dove troviamo una vetrina di videogiochi in vendita.
        Francesco ha un obiettivo specifico: trovare un gioco di simulazione. Per fare ciò si dirige verso il filtro di
        ricerca. Una volta attivato il filtro, il sistema ci mostra una lista di categorie tra cui scegliere. Francesco
        seleziona la categoria "simulazione" e conferma la scelta con la funzionalità "Applica filtro". Il sistema risponde
        aggiornando la pagina e mostrando un elenco di giochi appartenenti alla categoria "simulazione".
        Il sistema ha mostrato troppi videogames e Francesco non riesce a individuare il gioco che desidera tra le opzioni
        visualizzate. A questo punto, decide di utilizzare la funzione di "ricerca per nome", digita il nome del gioco
        desiderato: "GT7". Con un semplice tocco su "Cerca", il sistema elabora la richiesta e presenta i risultati.
        Tra questi, Francesco individua il gioco "GT7", visualizzandone l'immagine di copertina, una breve descrizione e il
        prezzo. Curioso di saperne di più sul gioco, Francesco chiede al sistema ulteriori informazioni tramite la funzionalità
        preposta "info gioco". In risposta, il sistema mostra ulteriori dettagli, tra cui l'età consigliata, le dimensioni
        del prodotto, la data di uscita e molto altro. Dopo aver esaminato attentamente le informazioni, decide di procedere
        all'acquisto e aggiunge il gioco al suo carrello virtuale utilizzando la funzionalità  "Aggiungi al carrello".
        Tuttavia, la sessione di shopping non finisce qui. Soddisfatto dell'acquisto, Francesco decide di tornare alla homepage
        per esplorare ulteriori prodotti e vedere cosa altro l'applicazione ha da offrire.

    \subsection*{S08: Acquisto cliente non registrato}
    \addcontentsline{toc}{subsection}{\protect\numberline{}S08: Acquisto cliente non registrato}
         L'utente Luca Neri accede alla piattaforma dal suo browser per la prima volta, avendone sentito parlare da alcuni conoscenti, entrando nella pagina principale.
         In particolare, rimane colpito da un videogioco presente tra i \textit{featured-products} e decide di acquistarlo. Dunque, per prima cosa, aggiunge il prodotto
         al carrello e si reca sulla pagina dedicata al carrello. Seleziona l'opzione di \textit{check-out}, ma, non avendo ancora un account, il sito lo reindirizza alla pagina
         di registrazione, dove compila i vari campi necessari alla registrazione: \textit{Nome: Luca}, \textit{Cognome: Neri}, \textit{Data di nascita: 27/05/1998},
         \textit{Codice Fiscale: LCUNRE98E27F205X}, \textit{Username: nlucas98}, \textit{Indirizzo e-mail: lneri98@gmail.com} e \textit{Password: LaMiaPasswordSafe12}. Infine, clicca
         su \textit{Registrati}. La registrazione termina con successo, e l'utente viene riportato sulla pagina del carrello. L'utente prosegue quindi al check-out e viene rimandato
         alla schermata per ultimare il pagamento.\\
         Compila i vari campi: \textit{Via: Via Piave}, \textit{Civico: 21}, \textit{Città: Voghera}, \textit{Provincia: PV}, \textit{Stato: Italia}, \textit{Cellulare: 327 352 3770},
         \textit{Numero di carta: 5333756908563312}, \textit{Scadenza: 12/24}, \textit{CVV: 550}. Infine, seleziona l'opzione per terminare l'acquisto.\\
         L'ordine viene acquisito dal sistema, che, verificati i dati relativi al pagamento, svuota il carrello dell'utente, il quale viene reindirizzato alla \textit{thank-you-page},
         dove gli verrà mostrato il riepilogo dell'ordine, con il relativo codice.

    \subsection*{S09: Acquisto cliente non loggato}
    \addcontentsline{toc}{subsection}{\protect\numberline{}S09: Acquisto cliente non loggato}
        L'utente \textit{Mario Rossi} accede alla pagina principale della piattaforma e si porta, attraverso l'apposita opzione, alla pagina
        del catalogo dalla quale sceglie due videogiochi e li aggiunge al carrello. Si reca, dunque, alla pagina del carrello, dove gli vengono mostrati i due prodotti presenti
        nel carrello, le opzioni per \textit{modificare la quantità} di ciascun prodotto e il \textit{totale}, che ammonta a \officialeuro 90,00. A questo punto, seleziona la funzionalità
        per effettuare il check-out e, non essendo loggato, viene reindirizzato alla pagina di \textit{login}. Una volta effettuato l'accesso, viene riportato dove aveva
        interrotto, cioè sulla pagina del carrello.
        Da qui, seleziona l'opzione di \textit{check-out}, che lo rimanda alla pagina di inserimento dei dati di spedizione e del metodo di pagamento. Compila i vari campi:
        \textit{Via: Via Carmine}, \textit{Civico: 57}, \textit{Città: Salerno}, \textit{Provincia: SA}, \textit{Stato: Italia}, \textit{Cellulare: 345 670 9276},
        \textit{Numero di carta: 5333547887933563}, \textit{Scadenza: 10/25}, \textit{CVV: 679}. Infine, seleziona l'opzione per terminare l'acquisto.\\
        L'ordine viene acquisito dal sistema, che, verificati i dati relativi al pagamento, svuota il carrello dell'utente, il quale viene
        reindirizzato alla \textit{thank-you-page}, dove gli verrà mostrato il riepilogo dell'ordine, con il relativo codice.

    \subsection*{S10: Acquisto cliente loggato}
    \addcontentsline{toc}{subsection}{\protect\numberline{}S10: Acquisto cliente loggato}
        L'utente \textit{Mario Rossi} accede alla pagina principale della piattaforma, seleziona l'opzione di accesso al suo account e viene reindirizzato alla pagina di \textit{login}.
        A questo punto, compila i campi del form \textit{username} e \textit{password} con i propri dati (username: marrossi, password: MyPassWord11) e lo conferma cliccando sull'opzione preposta.
        Viene, dunque, rimandato sulla home page, dalla quale sceglie un prodotto tra quelli in evidenza e lo aggiunge al carrello.\\
        Clicca successivamente sulla funzionalità del carrello e viene portato sulla pagina del carrello, dove \textit{incrementa} la quantità del prodotto da acquistare a 2.\\
        Sicuro del prodotto, della quantità che intende acquistare ed del prezzo finale, pari a \officialeuro 55, seleziona l'opzione di \textit{check-out}, che lo rimanda
        alla pagina di inserimento dei dati di spedizione e del metodo di pagamento. Compila i vari campi: \textit{Via: Via Carmine}, \textit{Civico: 57}, \textit{Città: Salerno}, \textit{Provincia: SA}, \textit{Stato: Italia}, \textit{Cellulare: 345 670 9276}, \textit{Numero di carta: 5333547887933563}, \textit{Scadenza: 10/25}, \textit{CVV: 679}. Infine, seleziona l'opzione per terminare l'acquisto.\\
        L'ordine viene acquisito dal sistema, che, verificati i dati relativi al pagamento, svuota il carrello dell'utente, il quale viene reindirizzato alla
        \textit{thank-you-page}, dove gli verrà mostrato il riepilogo dell'ordine, con il relativo codice.

    \subsection*{S11: Aggiornamento stato ordine}
    \addcontentsline{toc}{subsection}{\protect\numberline{}S11: Aggiornamento stato ordine}
        L'amministratore gestori Giovanni ha un nuovo collaboratore assegnato alla gestione ordini. Giovanni ha effettuato
        l'autenticazione al sistema e deve creare le credenziali per Luca. Accede alla funzionalità "gestore ordine",
        seleziona la funzionalità "crea nuovo gestore" ed il sistema mostra una pagina da compilare. Inserisce:
        \begin{itemize}
            \item [-] nome: Luca
            \item [-] cognome: Rossi
            \item [-] email: lucaRossi@gmail.it
            \item [-] città: Salerno
            \item [-] provincia: SA
            \item [-] via Vernieri, 89
        \end{itemize}
        ed effettua la registrazione. Il sistema mostra la password che dovrà utilizzare per l'accesso al sistema, questà
        potrà essere modificata in seguito da Luca che potrà sceglierne una a suo piacimento, sempre rispettando la formattazione
        prevista. Giovanni consegna username e password a Luca Rossi, il nuovo gestore degli ordini, che accede al sistema
        inserisce le proprie credenziali username : LucaRossi@gmail.it, seleziona il ruolo : gestore ordini ed inserisce la
        passwor per terminare l'autenticazione. Il sistema lo reindirizza alla sezione dedicata della gestione degli ordini,
        e elenca tutti gli ordini registrati. Al fine di semplificare il suo lavoro e focalizzarsi sugli ordini "pronti", Luca
        decide di applicare un filtro tramite l'apposita funzione. La tabella degli ordini viene quindi prontamente aggiornata
        per visualizzare solo gli ordini con lo stato "pronto". Tra questi ordini, Luca individua l'ordine con numero
        identificativo \#1234, che è stato recentemente spedito. Di conseguenza, Luca modifica lo stato di quest'ordine
        da "pronto" a "spedito". Luca prosegue il suo lavoro gestionale, aggiornando con successo altri cinque ordini,
        garantendo così che le informazioni siano sempre aggiornate e accurate. Successivamente, utilizza nuovamente la
        funzione di filtro per selezionare gli ordini con stato "spedito". Luca riceve la conferma della consegna
        dell'ordine \#4321, il quale transita dallo stato "spedito" a "consegnato". La giornata di lavoro di Luca continua
        con ulteriori aggiornamenti degli ordini, garantendo che ciascun ordine possa passare attraverso i vari stati
        possibili, tra cui "in preparazione","pronto", "spedito" e "consegnato."

    \subsection*{S12: Aggiunta di una recensione}
    \addcontentsline{toc}{subsection}{\protect\numberline{}S12: Aggiunta di una recensione}
        L'utente Marco Marroni ha un obiettivo ben preciso: recensire il gioco Assassin's Creed Mirage che ha appena acquistato.
        Prima di tutto, si autentica al sistema inserendo le proprie credenziali, ovvero Email: mmarroni, Password: Marc0marroni
        Il sistema avverte l'utente che l'autenticazione è andata a buon fine, dato che ha inserito i dati correttamente.
        L'utente accede all'area riservata del proprio account e successivamente utilizza la funzionalità
        che lo porta nella schermata dell'elenco degli ordini effettuati. La schermata presenta tutti gli ordini,
        da quelli completati a quelli ancora in spedizione. L'utente trova l'ordine in cui ha acquistato
        Assassin's Creed Mirage e accede alla sezione che lo permette di inserire la propria recensione
        in un campo di testo apposito, digitando il seguente messaggio: "Il prodotto è arrivato in condizioni perfette
        e lo consiglio a tutti gli appassionati!". L'utente utilizza la funzionalità che permette l'invio della recensione.
        Successivamente, il sistema avverte l'utente di attendere che la pubblicazione della sua recensione sia approvata
        da un gestore delle recensioni.

    \subsection*{S13: Approvazione di una recensione}
    \addcontentsline{toc}{subsection}{\protect\numberline{}S13: Pubblicazione di una recensione}
        Il gestore delle recensioni Luca Verdi, in una normale giornata di lavoro, deve controllare le varie richieste
        di recensioni (effettuate dagli utenti) che deve approvare e rendere pubbliche.
        Prima di tutto, il gestore entra nella sezione dell'autenticazione: inserisce le proprie credenziali, ovvero username: lverdi, Password: Lucaverd1
        e dopodiché spunta il ruolo di cui fa parte, ossia "Gestore delle recensioni".
        Il sistema avverte il gestore che l'autenticazione va a buon fine. Dopodiché, il gestore accede alla schermata
        Gestione Recensioni per visualizzare l'elenco di tutte le richieste di recensione. Il gestore nota che in cima
        alla lista è presente una recensione dell'utente Marco Marroni riferita al gioco Assassin's Creed Mirage,
        con il seguente messaggio: "Il prodotto è arrivato in condizioni perfette e lo consiglio a tutti gli appassionati!".
        Il gestore capisce che il testo è privo di contenuti offensivi e decide di approvare la pubblicazione della recensione.
        Il sistema avverte il gestore che la pubblicazione è avvenuta con successo.

    \subsection*{S14: Modifica Password}
    \addcontentsline{toc}{subsection}{\protect\numberline{}S14: Modifica Password}
        Il cliente Luca Neri è autenticato e si trova sulla homepage del sito web e desidera cambiare la sua password.
        Su questa pagina, Luca seleziona il pulsante "account".
        A questo punto, si apre un menu a tendina dal quale il cliente sceglie l'opzione "Visualizza Account".
        Il sistema reindirizza Luca alla pagina del suo account, dove sono visualizzati i dettagli del suo profilo
        Nome: Luca, Cognome: Neri, Data di nascita: 27/05/1998, Codice Fiscale: LCUNRE98E27F205X, Username: nlucas98,
        Indirizzo e-mail: lneri98@gmail.com e Password: LaMiaPasswordSafe1!.
        Sulla pagina del suo account, il cliente individua e seleziona l'opzione "Modifica Dati Account".
        Viene presentato un modulo di modifica dati account che include un campo per la modifica della password.
        Il cliente inserisce la sua nuova password: Luc!Gatto23 nel campo "Nuova Password" e la conferma nel campo "Conferma Password".
        Il sistema verifica la validità della nuova password in base ai criteri di sicurezza stabiliti.
        A questo punto conferma la modifica e visualizza un messaggio che lo informa del successo dell'operazione.

    \subsection*{S15: Visualizzazione prodotti}
    \addcontentsline{toc}{subsection}{\protect\numberline{}S15: Visualizzazione prodotti}
        Francesco effettua l'accesso al sistema con autenticazione inserisce  le sue credenziali d'accesso, username :
        "francesco@libero.it", seleziona il ruolo, inserisce la password "aaaabbb2" ed effettua il "Login". Il sistema non
        riconosce l'utente, segnala con il messaggio "utente non riconosciuto" e reindirizza Francesco alla pagina d'autenticazione.
        Francesco ripete il processo di autenticazione con meticolosità, prestando particolare attenzione alle specificità
        maiuscole/minuscole nella password (modificandola in "aaaAbbb2"). Questa volta, l'autenticazione ha esito positivo, il sistema
        riconosce Francesco come utente valido e lo indirizza alla homepage di iSnapGaming, dove lo accoglie con un messaggio di benvenuto.
        Francesco seleziona la funzionalità "I Miei Ordini", il sistema mostra una tabella che espone tutti gli ordini registrati nel sistema.
        Ogni riga della tabella contiene il numero d'ordine, una breve descrizione, il costo totale e lo stato corrente dell'ordine.
            \begin{itemize}
                \item [-] \#0567 Ghostrunner 2 	\officialeuro 40.98	consegnato
                \item [-] \#1678 GT7... \officialeuro 95.49	in lavorazione
            \end{itemize}
            Francesco seleziona l'ordine \#1678 per esaminare in dettaglio l'ordine ed il  sistema visualizza una seconda tabella
            che mostra quantità, nomi e prezzi di ciascun articolo dell'ordine:
            \begin{itemize}
                \setlength\itemsep{.05cm}
                \item [-] 1	GT7 \officialeuro 35.50
                \item [-] 1Assassin's Creed Mirage	\officialeuro 59.99
            \end{itemize}
        Alla conclusione della sua sessione d'interazione con iSnapGames, Francesco seleziona la funzione
        "Logout". In risposta, il sistema lo congeda riportandolo alla homepage generica di iSnapGaming.

\section*{Requirements}
\addcontentsline{toc}{section}{\protect\numberline{} Requirements}
    \subsection*{Functional requirements}
    \addcontentsline{toc}{section}{\protect\numberline{} Functional Requirements}
        \begin{itemize}
            \item FR\_01 (priorità: alta)
               \begin{itemize}
                   \item \textbf{Registrazione utente}: il sistema deve consentire all'utente di registrarsi.
                \end{itemize}
            \item FR\_02 (priorità: alta)
               \begin{itemize}
                   \item \textbf{Autenticazione utente}: il sistema deve consentire all'utente di autenticarsi.
                \end{itemize}
            \item FR\_03 (priorità: alta)
               \begin{itemize}
                   \item \textbf{Modifica dati account}: il sistema deve consentire all'utente di modificare i dati del proprio account.
                \end{itemize}
            \item FR\_04 (priorità: alta)
               \begin{itemize}
                   \item \textbf{Visualizzazione dati account}: il sistema deve consentire all'utente di visualizzare i dati relativi al proprio account.
                \end{itemize}
            \item FR\_05 (priorità: alta)
               \begin{itemize}
                   \item \textbf{Eliminazione account}: il sistema deve consentire all'utente di eliminare il proprio account.
                \end{itemize}
            \item FR\_06 (priorità: alta)
               \begin{itemize}
                   \item \textbf{Ricerca di un prodotto}: il sistema deve consentire all'utente di ricercare un prodotto.
                \end{itemize}
            \item FR\_07 (priorità: alta)
               \begin{itemize}
                   \item \textbf{Aggiunta di prodotti al carrello}: il sistema deve consentire all'utente di aggiungere un prodotto al carrello.
                \end{itemize}
            \newpage
            \item FR\_08 (priorità: alta)
               \begin{itemize}
                   \item \textbf{Visualizzazione contenuto del carrello}: il sistema deve consentire all'utente di visualizzare il contenuto del carrello.
                \end{itemize}
            \item FR\_09 (priorità: alta)
               \begin{itemize}
                   \item \textbf{Modifica quantità prodotti nel carrello}: il sistema deve consentire all'utente di modificare la quantità dei prodotti inseriti nel carrello.
                \end{itemize}
            \item FR\_10 (priorità: alta)
               \begin{itemize}
                   \item \textbf{Effettuazione acquisto}: il sistema deve consentire all'utente di effettuare un acquisto.
                \end{itemize}
            \item FR\_11 (priorità: bassa)
               \begin{itemize}
                   \item \textbf{Inserimento recensione}: il sistema deve consentire all'utente di aggiungere una recensione.
                \end{itemize}
            \item FR\_12 (priorità: bassa)
               \begin{itemize}
                   \item \textbf{Visualizzazione recensioni}: il sistema deve consentire all'utente di visualizzare le recensioni da lui effettuate.
                \end{itemize}
            \item FR\_13 (priorità: alta)
               \begin{itemize}
                   \item \textbf{Visualizzazione storico ordini}: il sistema deve consentire all'utente di visualizzare gli ordini da lui effettuati.
                \end{itemize}
            \item FR\_14 (priorità: alta)
               \begin{itemize}
                   \item \textbf{Aggiunta gestori}: il sistema deve consentire all'amministratore dei gestori di aggiungere un nuovo gestore.
                \end{itemize}
            \item FR\_15 (priorità: alta)
               \begin{itemize}
                   \item \textbf{Visualizzazione gestori}: il sistema deve consentire all'amministratore dei gestori di visualizzare gli account di ogni gestore.
                \end{itemize}
            \item FR\_16 (priorità: alta)
               \begin{itemize}
                   \item \textbf{Visualizzazione dati gestore}: il sistema deve consentire all'amministratore dei gestori di visualizzare i dati dell'account di un gestore.
                \end{itemize}
            \item FR\_17 (priorità: alta)
               \begin{itemize}
                   \item \textbf{Modificare dati gestore}: il sistema deve consentire all'amministratore dei gestori di modificare i dati dell'account di un gestore.
                \end{itemize}
            \item FR\_18 (priorità: alta)
               \begin{itemize}
                   \item \textbf{Eliminazione account gestore}: il sistema deve consentire all'amministratore dei gestori di eliminare l'account di un gestore.
                \end{itemize}
            \item FR\_19 (priorità: alta)
               \begin{itemize}
                   \item \textbf{Cambio ruolo gestore}: il sistema deve consentire ad un gestore di cambiare ruolo.
                \end{itemize}
            \item FR\_20 (priorità: alta)
               \begin{itemize}
                   \item \textbf{Aggiunta di una categoria}: il sistema deve consentire al gestore dei prodotti di aggiungere una nuova categoria.
                \end{itemize}
            \item FR\_21 (priorità: alta)
               \begin{itemize}
                   \item \textbf{Aggiunta di un prodotto}: il sistema deve consentire al gestore dei prodotti di inserire un nuovo prodotto.
                \end{itemize}
            \item FR\_22 (priorità: alta)
               \begin{itemize}
                   \item \textbf{Modifica di un prodotto}: il sistema deve consentire al gestore dei prodotti di modificare i dati di un prodotto.
                \end{itemize}
            \item FR\_23 (priorità: alta)
               \begin{itemize}
                   \item \textbf{Rimuovere un prodotto}: il sistema deve consentire al gestore dei prodotti di rimuovere un prodotto.
                \end{itemize}
            \item FR\_24 (priorità: alta)
               \begin{itemize}
                   \item \textbf{Visualizzazione ordini}: il sistema deve consentire al gestore degli ordini di visualizzare gli ordini effettuati dagli utenti.
                \end{itemize}
            \item FR\_25 (priorità: alta)
               \begin{itemize}
                   \item \textbf{Visualizzazione dettagli ordine}: il sistema deve consentire al gestore degli ordini di visualizzare i dettagli di uno specifico ordine.
                \end{itemize}
            \item FR\_26(priorità: alta)
               \begin{itemize}
                   \item \textbf{Aggiornamento stato ordine}: il sistema deve consentire al gestore degli ordini di aggiornare lo stato di un ordine.
                \end{itemize}
            \item FR\_27 (priorità: bassa)
               \begin{itemize}
                   \item \textbf{Visualizzazione recensioni}: il sistema deve consentire al gestore delle recensioni di visualizzare le recensioni inserite dagli utenti.
                \end{itemize}
            \item FR\_28 (priorità: bassa)
               \begin{itemize}
                   \item \textbf{Aggiornamento stato recensione}: il sistema deve consentire al gestore delle recensioni di aggiornare lo stato di una recensione.
                \end{itemize}
            \item FR\_29 (priorità: bassa)
               \begin{itemize}
                   \item \textbf{Eliminazione recensione}: il sistema deve consentire al gestore delle recensioni di eliminare una recensione.
                \end{itemize}
        \end{itemize} %END REQUISITI FUNZIONALI


    \subsection*{Non-functional requirements }
    \addcontentsline{toc}{section}{\protect\numberline{} Non-functional requirements}
        \subsubsection*{Dependability}
            \textbf{Security}
                \begin{itemize}
                    \item [-] \textbf{Prevenzione SQL injection}: Il sistema deve garantire la sicurezza e la protezione dei dati da attacchi di tipo SQL injection, assicurando che tutte le query SQL e i parametri siano correttamente validati e sanificati prima dell'esecuzione.
                    \item [-] \textbf{Crittografia password}: Il sistema deve implementare un meccanismo di crittografia forte e irreversibile per memorizzare le password degli utenti nel database. Le password devono essere gestite in modo che non sia possibile recuperarle in chiaro dal database.
                \end{itemize}
            \textbf{Robustness}
                \begin{itemize}
                    \item [-] \textbf{Gestione eccezioni}: Gestione delle eccezioni client-side. In caso di eccezioni e dati errati, il sistema dovrà essere in grado di gestirli e non dovrà andare in crash.
                \end{itemize}

        \subsubsection*{Usability}
            \begin{itemize}
                \item [-] \textbf{Design responsive}: Il sistema deve implementare un design responsive, in modo che l'interfaccia si adatti in maniera fluida a diverse dimensioni dello schermo, consentendo agli utenti di navigare e fare acquisti sia da desktop che da dispositivi mobile.
                \item [-] \textbf{Immagini di alta qualità}: Il sistema utilizzare immagini chiare e di alta qualità per favorire l'appeal dei prodotti e dell'intera applicazione.
                \item [-] \textbf{Selezione di colori}: Il sistema deve adottare una specifica palette di colori, definita come un insieme ben preciso di valori esadecimali nel formato \#RRGGBB, al fine di assicurare una chiara distinzione visuale tra le funzionalità di registrazione, acquisto e le altre funzioni principali dell'applicazione. I colori utilizzati dovrebbero essere selezionati in modo tale da consentire agli utenti di identificare rapidamente e senza ambiguità ciascuna funzionalità.
            \end{itemize}



\end{document}